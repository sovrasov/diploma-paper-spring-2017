\documentclass[aspectratio=1610]{beamer}

\usetheme{unnslides}

\usepackage{listings}
\usepackage{graphicx}
\usepackage{caption}
\usepackage{cmbright}
\usepackage{fontspec}
\usepackage{unicode-math}
\setmainfont{CMU Sans Serif}
\setromanfont{CMU Sans Serif}
\setsansfont{CMU Sans Serif}

\usepackage{polyglossia}
\setmainlanguage{russian}
\setbeamertemplate{itemize item}{\color{black}$\blacktriangleright$}

%set pages numeration
\setbeamertemplate{footline}[frame number]
\setbeamertemplate{headline}{}
\setlength\abovecaptionskip{-1pt}

\title{Исследование схем ускорения сходимости алгоритмов глобальной оптимизации}
\author{\textbf{В.В.~Соврасов}}
\institute{ННГУ им. Н.И. Лобачевского}
\date{}
\begin{document}

\begin{frame}[noframenumbering,plain]
\titlepage
\end{frame}

\begin{frame}
  \frametitle{Постановка задачи}
  \begin{displaymath}
    \begin{array}{lr}
      D=\{y\in R^N:a_i\leqslant x_i\leqslant{b_i}, 1\leqslant{i}\leqslant{N}\} \\
      \varphi(y^*)=\min\{\varphi(y):y\in D\}
    \end{array}
  \end{displaymath}
  Предполагается, что целевая функция удовлетворяет условию Липшица в области \(D\):
  \begin{displaymath}
  |\varphi(y_1)-\varphi(y_2)|\leqslant L\Vert y_1-y_2\Vert,y_1,y_2\in D,0<L<\infty
  \end{displaymath}
  Численное решение задачи означает построение оценки \(\widetilde{y}\), близкой по какой-либо
  норме к точке \(y^*\) на основе конечного числа значений целевой функции задачи,
  вычисленных в точках области \(D\).
\end{frame}

\begin{frame}
  \frametitle{Редукция размерности}
\end{frame}

\begin{frame}
  \frametitle{Метод глобальной оптимизации}
  Общая схема характеристического метода:
  пусть имеется \(k\) результатов испытаний, далее:

Шаг 1. Упорядочить поисковую информацию по возрастанию
координат точек испытаний

Шаг 2. Вычислить для каждого интервала
величину \(R(i)\), называемую характеристикой.

Шаг 3. Выбрать интервал номер \(t\) с наибольшей
характеристикой и провести в нем испытание:
  \begin{displaymath}
    x^{k+1}=d(t)\in (x_{t-1}, x_t)
  \end{displaymath}

  Критерий остановки:
  \begin{displaymath}
    \Vert x_t - x_{t-1}\Vert < \varepsilon
  \end{displaymath}
\end{frame}

\begin{frame}
  \frametitle{Метод глобальной оптимизации}
\end{frame}

\begin{frame}
  \frametitle{Класс тестовых задач}
\end{frame}

\begin{frame}
\frametitle{Использование методов локальной оптимизации}
Способы использования локального поиска (метод Хука-Дживса):\
\begin{enumerate}
  \item Запуск из лучшей найденной точки после окончания работы АГП.
  \item Запуски из текущих лучших точек в процессе работы АГП.
\end{enumerate}
\bigbreak
Стратегии сохранения информации (для п. 2):
\begin{itemize}
  \item добавлять только лучшие точки
  \item добавлять в поисковую информацию все точки
\end{itemize}
\end{frame}

\begin{frame}
  \frametitle{Использование методов локальной оптимизации}
  Результаты применения различных стратегий сохранения информации:
  \begin{figure}[ht]
        \begin{minipage}[b]{0.49\linewidth}
            \centering
            \includegraphics[width=\textwidth]{../paper/images/local_search_op.png}
            \caption*{GKLS 4d Simple}
        \end{minipage}
        \begin{minipage}[b]{0.49\linewidth}
            \centering
            \includegraphics[width=\textwidth]{../paper/images/local_search_op.png}
            \caption*{GKLS 5d Simple}
        \end{minipage}
    \end{figure}
\end{frame}

\begin{frame}
\frametitle{Смешанный алгоритм}
Метод является модификацией АГП. Каждый интервал имеет две
характеристики \(R(i)\) и \(R^*(i)\).
  \begin{displaymath}
  R^*(i)=\frac{R(i)}{\sqrt{(z_i-z^*)(z_{i-1}-z^*)}/\mu + 1.5^{-\alpha}}
  \end{displaymath}
  Для эффективной реализации АГП используется приоритетная
  очередь интрервалов. Ключ – \(R(i)\).
  Для смешанного АГП – две связанные очереди.
  Операции с очередями:
  \begin{itemize}
    \item Синхронная вставка
    \item Синхронное удаление
    \item Обновление перекрестных ссылок при восстановлении кучеобразности
  \end{itemize}
\end{frame}

\defverbatim[colored]\lstI{
\begin{lstlisting}[language=C++,keywordstyle=\color{red}]
inline void swapElems(T& arg1, T& arg2)
{
  if (arg1.pLinkedElement != NULL &&
      arg2.pLinkedElement != NULL)
    std::swap(arg1.pLinkedElement->pLinkedElement,
      arg2.pLinkedElement->pLinkedElement);
  else if (arg1.pLinkedElement != NULL)
    arg1.pLinkedElement->pLinkedElement = &arg2;
  else if(arg2.pLinkedElement != NULL)
    arg2.pLinkedElement->pLinkedElement = &arg1;
  std::swap(arg1, arg2);
}
\end{lstlisting}
}

\begin{frame}
\frametitle{Смешанный алгоритм}
Схема перекрёстных ссылок на элементы очередей:
    \begin{figure}
      \center
        \includegraphics[width=.7\textwidth]{../paper/images/examin_heaps.png}
    \end{figure}
\end{frame}

\begin{frame}
\frametitle{Смешанный алгоритм}
Алгоритм обмена элементов при погружении/всплытии:
\lstI
\end{frame}

\begin{frame}
  \frametitle{Смешанный алгоритм}
  \begin{figure}
    \center
      \includegraphics[width=0.7\textwidth]{../paper/images/mixed_op4d.png}
      \caption*{Операционные характеристики обычного и смешанного АГП на классе GKLS 4d Simple}
  \end{figure}
\end{frame}

\end{document}
