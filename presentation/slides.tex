\documentclass{beamer}

\usetheme{unnslides}

\usepackage{cmbright}

\usepackage{fontspec}
\setmainfont{CMU Sans Serif}
\setromanfont{CMU Sans Serif}
\setsansfont{CMU Sans Serif}

\usepackage{polyglossia}
\setmainlanguage{russian}
\setbeamertemplate{itemize item}{\color{black}$\blacktriangleright$}

\title{Исследование схем ускорения сходимости алгоритмов глобальной оптимизации}
\author{\textbf{В.В.~Соврасов}}
\institute{ННГУ им. Н.И. Лобачевского}
\date{}

\begin{document}

\begin{frame}
\titlepage
\end{frame}

\begin{frame}
\frametitle{Метод частиц в ячейках в PICADOR}
    \begin{figure}
        %\includegraphics[width=\textwidth]{img/pic}
    \end{figure}
\end{frame}

\begin{frame}
\frametitle{Смешанный алгоритм глобального поиска и его эффективная реализация}
Метод является модификацией АГП. Каждый интервал имеет две
характеристики \(R(i)\) и \(R^*(i)\).
  \begin{displaymath}
  R^*(i)=\frac{R(i)}{\sqrt{(z_i-z^*)(z_{i-1}-z^*)}/\mu + 1.5^{-\alpha}}
  \end{displaymath}
  Для эффективной реализации АГП используется приоритетная
  очередь интрервалов. Ключ – \(R(i)\).
  Для смешанного АГП – две связанные очереди.
  Операции с очередями:
  \begin{itemize}
    \item Синхронная вставка
    \item Синхронное удаление
    \item Обновление перекрестных ссылок при восстановлении кучеобразности
  \end{itemize}
\end{frame}

\begin{frame}
\frametitle{Использование методов локальной оптимизации}

    \begin{figure}
        %\includegraphics[width=0.8\textwidth]{../poster/img/scheme1}
    \end{figure}
\end{frame}

\end{document}
