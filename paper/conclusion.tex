\section{Заключение}
В ходе работы были получены следующие практические результаты:
\begin{itemize}
  \item реализован метод локальной оптимизации Хука-Дживса (код в приложении \ref{attach3});
  \item в системе Globalizer реализованы различные стратегии использования локального поиска,
  исследована их эффективность;
  \item в рамках Globalizer реализована поддержка смешанного алгоритма глобального поиска,
  а также эффективные структуры данных, необходимые для этого алгоритма (фрагменты кода в приложении \ref{attach4})
  \item был проведён отдельный эксперимент для выяснения возможности практического использования многоуровневых развёрток.
  \item рассмотрено применение ранее предложенного для одномерных методов способа учёта
  локального поведения оптимизируемой функции в методе многомерной многоэкстремальной
  оптимизации. Учёт локальных свойств выражается в использовании различных оценок
  константы Гёльдера в различных областях поиска. Эффективность рассматриваемого подхода
  была подтверждена решением существенно многоэкстремальных задач из двух тестовых классов.
  Реализация модифицированного алгоритма глобального поиска на языке C++ представлена
  в приложении \ref{attach1};
  \item реализована схема вычисления целевой функции в задаче поиска оптимального управления,
  описанной в разделе \ref{sec:optimal_cpntrol}
  При реализации использовадась библиотека Eigen \cite{eigenLib} для решения задач
  линейной алгебры. Исходный код можно найти в приложении \ref{attach2}
  Два варианта задачи поиска оптимального управления,
  приведённые в \cite{optControl}, решены с помощью системы Globalizer;
  \item решена прикладная задача построения Парето-границы в одной задаче поиска оптимального
  управления.
\end{itemize}
