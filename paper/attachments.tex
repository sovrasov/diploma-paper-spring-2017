\section{Приложения}
\subsection{Приложение 1}
\label{attach1}

\begin{lstlisting}[frame=single]
#ifndef OPTIMIZER_ALGORITHM_UNCONSTRAINED_HPP
#define OPTIMIZER_ALGORITHM_UNCONSTRAINED_HPP

#include "OptimizerCoreGlobal.hpp"
#include "OptimizerTask.hpp"
#include "OptimizerSolution.hpp"
#include "OptimizerFunction.hpp"
#include "OptimizerDataStructures.hpp"
#include "OptimizerSolution.hpp"
#include "OptimizerResult.hpp"
#include "OptimizerSearchSequence.hpp"

#include <set>

namespace optimizercore
{

  class EXPORT_API OptimizerAlgorithmUnconstrained final
  {

  private:

    bool mLocalMixType;
    bool mIsAlgorithmMemoryAllocated;
    bool mIsParamsInitialized;
    bool mIsTaskInitialized;
    bool mNeedLocalVerification;

    int mNumberOfThreads;
    int mLocalStartIterationNumber;
    int mMaxNumberOfIterations;
    int mMapTightness;
    int mMethodDimension;
    int mAlpha;
    int mLocalMixParameter;
    int mMapType;

    OptimizerSpaceTransformation mSpaceTransform;
    OptimizerFunction *mTargetFunction;
    OptimizerFunctionPtr mTargetFunctionSmartPtr;

    OptimizerInterval *mIntervalsForTrials;
    std::set<OptimizerTrialPoint> mSearchInformationStorage;
    OptimizerTrialPoint mOptimumEvaluation, *mNextTrialsPoints;

    LocalTuningMode mLocalTuningMode;

    double mGlobalM, mZ, eps, r, mMaxIntervalNorm;
    double **mNextPoints;

    void AllocMem();
    void InitializeInformationStorage();
    void UpdateGlobalM(std::set<OptimizerTrialPoint>::iterator&);
    int UpdateRanks(bool isLocal);
    bool InsertNewTrials(int trailsNumber);
    OptimizerSolution DoLocalVerification(OptimizerSolution startPoint);

  public:
    OptimizerAlgorithmUnconstrained();
    ~OptimizerAlgorithmUnconstrained();

    void SetTask(OptimizerFunctionPtr function,
      OptimizerSpaceTransformation spaceTransform);
    void SetThreadsNum(int num);
    void SetParameters(OptimizerParameters params);

    OptimizerResult StartOptimization(const double* xOpt,
      StopCriterionType stopType);

    double GetLipschitzConst() const;
    OptimizerSearchSequence GetSearchSequence() const;

  };
}
#endif
\end{lstlisting}
\begin{lstlisting}[frame=single]
  #include "OptimizerAlgorithmUnconstrained.hpp"
  #include "HookeJeevesLocalMethod.hpp"

  #include <cassert>
  #include <algorithm>

  using namespace optimizercore;
  using namespace optimizercore::utils;

  OptimizerAlgorithmUnconstrained::OptimizerAlgorithmUnconstrained()
  {
    mIsAlgorithmMemoryAllocated = false;

    mLocalStartIterationNumber = 1;
    mNumberOfThreads = 1;
    mMaxNumberOfIterations = 5000;
    mNextPoints = nullptr;
    mNextTrialsPoints = nullptr;
    mIntervalsForTrials = nullptr;
    r = 2;
    mLocalTuningMode = LocalTuningMode::None;

    mIsTaskInitialized = false;
    mIsParamsInitialized = false;
  }

  void OptimizerAlgorithmUnconstrained::SetTask(OptimizerFunctionPtr function,
    OptimizerSpaceTransformation spaceTransform)
  {
    assert(function);

    mTargetFunctionSmartPtr = function;
    mTargetFunction = function.get();
    mSpaceTransform = spaceTransform;

    mIsTaskInitialized = true;
  }

  OptimizerSearchSequence OptimizerAlgorithmUnconstrained::GetSearchSequence() const
  {
    return OptimizerSearchSequence(mSearchInformationStorage, mMethodDimension,
      static_cast<MapType> (mMapType), mMapTightness, mSpaceTransform);
  }

  double OptimizerAlgorithmUnconstrained::GetLipschitzConst() const
  {
    return mGlobalM;
  }

  void OptimizerAlgorithmUnconstrained::SetParameters(OptimizerParameters params)
  {
    assert(params.algDimention);
    assert(params.eps > 0);
    assert(params.localAlgStartIterationNumber > 0);
    assert(params.mapTightness > 5 && params.mapTightness <= 20);
    assert(params.maxIterationsNumber > 0);
    assert(params.localMixParameter >= 0 && params.localMixParameter <= 20);
    assert(params.r != nullptr);
    assert(params.numberOfThreads > 0);
    assert(params.reserves != nullptr);
    assert(params.numberOfMaps > 0);

    mLocalStartIterationNumber = params.localAlgStartIterationNumber;
    eps = params.eps;
    if (params.localMixParameter <= 10)	{
      mLocalMixParameter = params.localMixParameter;
      mLocalMixType = true;
    }
    else	{
      mLocalMixParameter = 20 - params.localMixParameter;
      mLocalMixType = false;
    }
    mNeedLocalVerification = params.localVerification;
    mAlpha = params.localExponent;
    mMethodDimension = params.algDimention;
    mMapTightness = params.mapTightness;
    mMapType = static_cast<int>(params.mapType);
    mMaxNumberOfIterations = params.maxIterationsNumber;
    mLocalTuningMode = params.localTuningMode;
    r = *params.r;
    if (mNextPoints)
      utils::DeleteMatrix(mNextPoints, mNumberOfThreads);
    mNextPoints = utils::AllocateMatrix<double>(mNumberOfThreads, mMethodDimension);
    this->SetThreadsNum(params.numberOfThreads);

    mIsParamsInitialized = true;
  }

  void OptimizerAlgorithmUnconstrained::InitializeInformationStorage()
  {
    if (!mIsAlgorithmMemoryAllocated){
      AllocMem();
      mIsAlgorithmMemoryAllocated = true;
    }

    mZ = HUGE_VAL;
    mGlobalM = 1;
    mMaxIntervalNorm = 0;

    mSearchInformationStorage.clear();

    mapd(0.0, mMapTightness, mNextPoints[0], mMethodDimension, mMapType);
    mSpaceTransform.Transform(mNextPoints[0], mNextPoints[0]);
    mSearchInformationStorage.emplace(0.0, mTargetFunction->Calculate(mNextPoints[0]), 0);

    mapd(1.0, mMapTightness, mNextPoints[0], mMethodDimension, mMapType);
    mSpaceTransform.Transform(mNextPoints[0], mNextPoints[0]);
    mSearchInformationStorage.emplace(1.0, mTargetFunction->Calculate(mNextPoints[0]), 0);
  }

  bool OptimizerAlgorithmUnconstrained::InsertNewTrials(int trailsNumber)
  {
    bool storageInsertionError;
    if (mMapType == 3)
    {
      int preimagesNumber = 0;
      double preimages[32];
      for (int i = 0; i < trailsNumber; i++)
      {
        invmad(mMapTightness, preimages, 32,
          &preimagesNumber, mNextPoints[i], mMethodDimension, 4);
        for (int k = 0; k < preimagesNumber; k++)
        {
          mNextTrialsPoints[i].x = preimages[k];
          auto insertionResult =
            mSearchInformationStorage.insert(mNextTrialsPoints[i]);

          if (!(storageInsertionError = insertionResult.second))
            break;

          UpdateGlobalM(insertionResult.first);
        }
      }
    }
    else
      for (int i = 0; i < trailsNumber; i++)
      {
        auto insertionResult =
          mSearchInformationStorage.insert(mNextTrialsPoints[i]);

        if (!(storageInsertionError = insertionResult.second))
          break;

        UpdateGlobalM(insertionResult.first);
      }
    return storageInsertionError;
  }

  OptimizerResult OptimizerAlgorithmUnconstrained::StartOptimization(
    const double* a, StopCriterionType stopType)
  {
    assert(mIsParamsInitialized && mIsTaskInitialized);
    assert(mSpaceTransform.GetDomainDimension() == mMethodDimension);

    InitializeInformationStorage();

    double *y;
    bool stop = false;
    int iterationsCount = 0,
      currentThrNum = 1, ranksUpdateErrCode;

    mNextTrialsPoints[0].x = 0.5;
    mapd(mNextTrialsPoints[0].x, mMapTightness, mNextPoints[0],
      mMethodDimension, mMapType);
    mSpaceTransform.Transform(mNextPoints[0], mNextPoints[0]);

    while (iterationsCount < mMaxNumberOfIterations && !stop)	{
      iterationsCount++;

  #pragma omp parallel for num_threads(currentThrNum)
      for (int i = 0; i < currentThrNum; i++)	{
        mNextTrialsPoints[i].val = mTargetFunction->Calculate(mNextPoints[i]);
        if (mMapType == 3)
          mSpaceTransform.InvertTransform(mNextPoints[i], mNextPoints[i]);
  #pragma omp critical
        if (mNextTrialsPoints[i].val < mZ)
          mZ = mNextTrialsPoints[i].val;
      }

      if (!InsertNewTrials(currentThrNum))
        break;

      if (iterationsCount >= mLocalStartIterationNumber)	{
        if (iterationsCount % (12 - mLocalMixParameter) == 0
          && mLocalMixParameter > 0)
          ranksUpdateErrCode = UpdateRanks(mLocalMixType);
        else
          ranksUpdateErrCode = UpdateRanks(!mLocalMixType);
      }
      else
        ranksUpdateErrCode = UpdateRanks(false);

      if (iterationsCount >= mNumberOfThreads + 10)
        currentThrNum = mNumberOfThreads;

      for (int i = 0; i < currentThrNum && !stop; i++)	{
        OptimizerTrialPoint left = mIntervalsForTrials[i].left;
        OptimizerTrialPoint right = mIntervalsForTrials[i].right;

        mNextTrialsPoints[i].x = (left.x + right.x) / 2
          - sgn(right.val - left.val)*pow(fabs(right.val - left.val)
          / mIntervalsForTrials[i].localM, mMethodDimension) / (2 * r);

        mapd(mNextTrialsPoints[i].x, mMapTightness, mNextPoints[i],
          mMethodDimension, mMapType);
        mSpaceTransform.Transform(mNextPoints[i], mNextPoints[i]);

        y = mNextPoints[i];

        if (stopType == StopCriterionType::OptimalPoint)	{
          if (NormNDimMax(y, a, mMethodDimension) < eps)	{
            stop = true;
            mOptimumEvaluation = mNextTrialsPoints[i];
          }
        }
        else	{
          if (pow(right.x - left.x, 1.0 / mMethodDimension) < eps)	{
            stop = true;
            mOptimumEvaluation = mNextTrialsPoints[i];
          }
        }
      }
    }

    mOptimumEvaluation.val = mTargetFunction->Calculate(y);
    mSearchInformationStorage.insert(mOptimumEvaluation);

    if (stopType == StopCriterionType::Precision)
      mOptimumEvaluation = *std::min_element(mSearchInformationStorage.begin(),
        mSearchInformationStorage.cend(),
        [](OptimizerTrialPoint p1, OptimizerTrialPoint p2)
      {
        return p1.val < p2.val;
      });

    mapd(mOptimumEvaluation.x, mMapTightness, y, mMethodDimension, mMapType);
    mSpaceTransform.Transform(y, y);

    SharedVector optPoint(new double[mMethodDimension], array_deleter<double>());
    std::memcpy(optPoint.get(), y, mMethodDimension*sizeof(double));

    OptimizerSolution solution(iterationsCount, mOptimumEvaluation.val,
      mOptimumEvaluation.x, mMethodDimension, optPoint);

    if (mNeedLocalVerification)
      return OptimizerResult(DoLocalVerification(solution));
    else
      return OptimizerResult(solution);
  }
  OptimizerSolution OptimizerAlgorithmUnconstrained::DoLocalVerification(OptimizerSolution startSolution)
  {
    OptimizerFunctionPtr *functions = new OptimizerFunctionPtr[1];
    functions[0] = mTargetFunctionSmartPtr;

    OptimizerTask localTask(std::shared_ptr<OptimizerFunctionPtr>(functions,
      utils::array_deleter<OptimizerFunctionPtr>()),
      0, mMethodDimension, mSpaceTransform.GetLeftDomainBound(),
      mSpaceTransform.GetRightDomainBound());

    localoptimizer::HookeJeevesLocalMethod localMethod;
    localMethod.SetEps(eps / 100);
    localMethod.SetInitialStep(2 * eps);
    localMethod.SetProblem(localTask);
    localMethod.SetStepMultiplier(2);
    localMethod.SetStartPoint(startSolution.GetOptimumPoint().get(),
      localTask.GetTaskDimension());

    SharedVector localOptimum(new double[mMethodDimension], array_deleter<double>());
    localMethod.StartOptimization(localOptimum.get());
    double bestLocalValue = mTargetFunction->Calculate(localOptimum.get());

    if (startSolution.GetOptimumValue() > bestLocalValue)
      return OptimizerSolution(startSolution.GetIterationsCount(),
      bestLocalValue, 0.5, mMethodDimension, localOptimum);

    return startSolution;
  }
  void OptimizerAlgorithmUnconstrained::SetThreadsNum(int num)
  {
    if (num > 0 && num < 100)
    {
      if (mNextPoints != nullptr)
        utils::DeleteMatrix(mNextPoints, mNumberOfThreads);
      mNumberOfThreads = num;
      if (mNextTrialsPoints)
        delete[] mNextTrialsPoints;
      if (mIntervalsForTrials)
        delete[] mIntervalsForTrials;
      mIntervalsForTrials = new OptimizerInterval[num];
      mNextTrialsPoints = new OptimizerTrialPoint[num];
      mNextPoints = utils::AllocateMatrix<double>(
        mNumberOfThreads, mMethodDimension);
    }
  }
  OptimizerAlgorithmUnconstrained::~OptimizerAlgorithmUnconstrained()
  {
    if (mIntervalsForTrials)
      delete[] mIntervalsForTrials;
    if (mNextPoints)
      utils::DeleteMatrix(mNextPoints, mNumberOfThreads);
    if (mNextTrialsPoints)
      delete[] mNextTrialsPoints;
    if (mIsAlgorithmMemoryAllocated)
    {
    }
  }
  void OptimizerAlgorithmUnconstrained::UpdateGlobalM(
    std::set<OptimizerTrialPoint>::iterator& newPointIt)
  {
    double max = mGlobalM;
    if (max == 1) max = 0;


    auto leftPointIt = newPointIt;
    auto rightPointIt = newPointIt;
    --leftPointIt;
    ++rightPointIt;

    double leftIntervalNorm = pow(newPointIt->x - leftPointIt->x, 1.0 / mMethodDimension);
    double rightIntervalNorm = pow(rightPointIt->x - newPointIt->x, 1.0 / mMethodDimension);


    max = fmax(fmax(fabs(newPointIt->val - leftPointIt->val) / leftIntervalNorm,
      fabs(rightPointIt->val - newPointIt->val) /	rightIntervalNorm), max);


    mMaxIntervalNorm = 0;
    auto currentPointIt = mSearchInformationStorage.begin();
    auto nextPointIt = currentPointIt;
    ++nextPointIt;

    while (nextPointIt != mSearchInformationStorage.cend())
    {
      if (mLocalTuningMode != LocalTuningMode::None)
        mMaxIntervalNorm = fmax(
          pow(nextPointIt->x - currentPointIt->x, 1.0 / mMethodDimension),
          mMaxIntervalNorm);

      ++currentPointIt;
      ++nextPointIt;
    }
    if (max != 0)
      mGlobalM = max;
    else
      mGlobalM = 1;
  }
  int OptimizerAlgorithmUnconstrained::UpdateRanks(bool isLocal)
  {
    double dx, curr_rank, mu1 = -HUGE_VAL, localM = mGlobalM;
    double localMConsts[3];

    for (int i = 0; i < mNumberOfThreads; i++)
      mIntervalsForTrials[i].R = -HUGE_VAL;

    auto leftIt = mSearchInformationStorage.begin();
    auto rightIt = mSearchInformationStorage.begin();
    ++rightIt;

    int storageSize = mSearchInformationStorage.size();

    for (int j = 0; j < storageSize - 1; j++)
    {
      dx = pow(rightIt->x - leftIt->x, 1.0 / mMethodDimension);

      if (dx == 0)
        return 1;

      if (mLocalTuningMode != LocalTuningMode::None)	{
        std::set<OptimizerTrialPoint>::iterator rightRightIt = rightIt;

        if (j > 0 && j < storageSize - 2)	{
          ++rightRightIt;

          std::swap(localMConsts[0], localMConsts[1]);
          std::swap(localMConsts[1], localMConsts[2]);

          localMConsts[2] = fabs(rightRightIt->val - rightIt->val)
            / pow(rightRightIt->x - rightIt->x, 1.0 / mMethodDimension);

          mu1 = fmax(fmax(localMConsts[0], localMConsts[1]), localMConsts[2]);
        }
        else if (j == 0)	{
          ++rightRightIt;

          localMConsts[1] = fabs(rightIt->val - leftIt->val) / dx;
          localMConsts[2] = fabs(rightRightIt->val - rightIt->val) /
            pow(rightRightIt->x - rightIt->x, 1.0 / mMethodDimension);
          mu1 = fmax(localMConsts[1], localMConsts[2]);
        }
        else
          mu1 = fmax(localMConsts[1], localMConsts[2]);

        double mu2 = mGlobalM*dx / mMaxIntervalNorm;

        if (mLocalTuningMode == LocalTuningMode::Maximum)	{
          localM = fmax(fmax(mu1, mu2), 0.01);
        }
        else// LocalTuningMode::Adaptive
          localM = fmax(mu1 / r + (1 - 1 / r)*mGlobalM, 0.01);
          //localM = fmax(mu1*(1 - dx / mMaxIntervalNorm) + mu2, 0.01);
          //localM = fmax(mu1*mMConvolution + (1 - mMConvolution)*mu2, 0.01);
      }

      curr_rank = dx + Pow2((rightIt->val - leftIt->val) / (r * localM)) / dx
        - 2 * (rightIt->val + leftIt->val - 2 * mZ) / (r * localM);
      if (isLocal)
        curr_rank /= sqrt((rightIt->val - mZ)*
        (leftIt->val - mZ)) / localM + pow(1.5, -mAlpha);

      if (curr_rank > mIntervalsForTrials[mNumberOfThreads - 1].R)
      {
        OptimizerInterval newInterval(
          OptimizerTrialPoint(*leftIt),
          OptimizerTrialPoint(*rightIt), curr_rank, localM);
        for (int i = 0; i < mNumberOfThreads; i++)
          if (mIntervalsForTrials[i].R < newInterval.R)
            std::swap(mIntervalsForTrials[i], newInterval);
      }
      ++leftIt;
      ++rightIt;
    }
    return 0;
  }
  void OptimizerAlgorithmUnconstrained::AllocMem()
  {
  }
\end{lstlisting}

\subsection{Приложение 2}
\label{attach2}
\begin{lstlisting}[frame=single]
#ifndef __OPTMAL_CONTROL_PROBLEM_H__
#define __OPTMAL_CONTROL_PROBLEM_H__

#include "problem_interface.h"
#include "optimal_problem_base.h"

#include <Eigen/Dense>
#include <string>

class TOptimalControlProblem : public IProblem
{
protected:

  OptimalControlProblemBase* mPProblemImpl;
  int mDimension;
  bool mIsInitialized;
  std::string mConfigPath;

public:
  TOptimalControlProblem();
  virtual int SetConfigPath(const std::string& configPath);
  virtual int SetDimension(int dimension);
  virtual int GetDimension() const;
  virtual int Initialize();

  virtual void GetBounds(double* lower, double *upper);
  virtual int GetOptimumValue(double& value) const;
  virtual int GetOptimumPoint(double* x) const;

  virtual int GetNumberOfFunctions() const;
  virtual int GetNumberOfConstraints() const;
  virtual int GetNumberOfCriterions() const;

  virtual double CalculateFunctionals(const double* x, int fNumber);

  ~TOptimalControlProblem();
};

extern "C" LIB_EXPORT_API IProblem* create();
extern "C" LIB_EXPORT_API void destroy(IProblem* ptr);
#endif
\end{lstlisting}
\begin{lstlisting}[frame=single]
#include "optimalControl.h"
#include "test_problems.h"
#include "problemA.h"
#include "problemB.h"
#include "pugixml.hpp"

#include <string>
#include <stdexcept>

TOptimalControlProblem::TOptimalControlProblem()
{
  mIsInitialized = false;
}

int TOptimalControlProblem::SetConfigPath(const std::string& configPath)
{
  mConfigPath = std::string(configPath);
  return IProblem::OK;
}

int TOptimalControlProblem::SetDimension(int dimension)
{
    return IProblem::OK;
}

int TOptimalControlProblem::GetDimension() const
{
  return mDimension;
}

int TOptimalControlProblem::Initialize()
{
  if (mIsInitialized == false)
  {
    mIsInitialized = true;

    pugi::xml_document doc;
    pugi::xml_parse_result result = doc.load_file(mConfigPath.c_str());
    if (result.status != pugi::status_ok)
      return IProblem::ERROR;

    pugi::xml_node config = doc.child("config");
    std::string problemName = config.child("problem_name").child_value();
    double secondCriterionLevel = 0.;
    double lambda = 0.;
    int dimension = 0;
    try  {
      secondCriterionLevel = std::stod(config.child("S").child_value());
      lambda = std::stod(config.child("lambda").child_value());
      dimension = std::stoi(config.child("N").child_value());
    }
    catch (std::invalid_argument& exp)  {
      return IProblem::ERROR;
    }

    if (problemName == std::string("v"))
      mPProblemImpl = new VibroisolationProblem();
    else if (problemName == std::string("od"))
      mPProblemImpl = new OscillationDampingProblem();
    else if (problemName == std::string("a2d"))
      mPProblemImpl = new ProblemA2d(secondCriterionLevel);
    else if (problemName == std::string("a3d"))
      mPProblemImpl = new ProblemA3d(secondCriterionLevel);
    else if (problemName == std::string("bNd") && dimension == 2)
      mPProblemImpl = new ProblemB2d(secondCriterionLevel);
    else if (problemName == std::string("bNd"))
      mPProblemImpl = new ProblemB(secondCriterionLevel, dimension);
    else if (problemName == std::string("bNdC"))
      mPProblemImpl = new ProblemBConvolved(lambda, dimension);
    else
      return IProblem::ERROR;

    mDimension = mPProblemImpl->GetDimension();

    return IProblem::OK;
  }
  else
    return IProblem::ERROR;
}

void TOptimalControlProblem::GetBounds(double* lower, double *upper)
{
  if (mIsInitialized)
    mPProblemImpl->GetBounds(lower, upper);
}

int TOptimalControlProblem::GetOptimumValue(double& value) const
{
  return IProblem::UNDEFINED;
}

int TOptimalControlProblem::GetOptimumPoint(double* point) const
{
  return IProblem::UNDEFINED;
}

int TOptimalControlProblem::GetNumberOfFunctions() const
{
  return mPProblemImpl->GetNumberOfFunctions();
}

int TOptimalControlProblem::GetNumberOfConstraints() const
{
  return mPProblemImpl->GetNumberOfConstraints();
}

int TOptimalControlProblem::GetNumberOfCriterions() const
{
  return 1;
}

// ------------------------------------------------------------------------------------------------
double TOptimalControlProblem::CalculateFunctionals(const double* x, int fNumber)
{
  return mPProblemImpl->CalculateFunctionals(x, fNumber);
}

TOptimalControlProblem::~TOptimalControlProblem()
{
  if(mIsInitialized)
  {
    mPProblemImpl->PrintFinalMessage();
    delete mPProblemImpl;
  }
}

LIB_EXPORT_API IProblem* create()
{
  return new TOptimalControlProblem();
}

LIB_EXPORT_API void destroy(IProblem* ptr)
{
  delete ptr;
}
\end{lstlisting}
\begin{lstlisting}[frame=single]
#ifndef OPTIMAL_CONTROL_BASE_H
#define OPTIMAL_CONTROL_BASE_H

#ifndef EIGEN_DONT_PARALLELIZE
#define EIGEN_DONT_PARALLELIZE
#endif

#include <Eigen/Dense>
#include <vector>

class OptimalControlProblemBase
{
protected:
  int mDimension;
  int n_x;
  int n_v;
  int n_u;
  std::vector<int> n_k;

  Eigen::MatrixXd A;
  Eigen::MatrixXd B_u;
  Eigen::MatrixXd B_v;
  Eigen::MatrixXd* YVecs;
  std::vector<Eigen::MatrixXd> CMatrices;
  std::vector<Eigen::MatrixXd> DMatrices;

  double mZeroConstraintOffset;
  double mS;

  virtual Eigen::RowVectorXd getTheta(const double* x);
  virtual double CalculateCriterionValue(const Eigen::RowVectorXd& theta, int fNumber);

public:
  OptimalControlProblemBase();
  virtual ~OptimalControlProblemBase();
  virtual void GetBounds(double* lower, double *upper) const = 0;
  virtual double CalculateFunctionals(const double* x, int fNumber);
  int GetDimension() const { return mDimension; }
  virtual int GetNumberOfConstraints() const { return 2; }
  virtual int GetNumberOfFunctions() const { return 3; }
  virtual void PrintFinalMessage() {}
};

#endif
\end{lstlisting}
\begin{lstlisting}[frame=single]
#include "optimal_problem_base.h"

#define _USE_MATH_DEFINES
#include <math.h>
#include <algorithm>
#include <omp.h>

using namespace Eigen;

OptimalControlProblemBase::OptimalControlProblemBase()
{
  YVecs = new MatrixXd[omp_get_num_procs()];
}

OptimalControlProblemBase::~OptimalControlProblemBase()
{
  delete[] YVecs;
}

RowVectorXd OptimalControlProblemBase::getTheta(const double* x)
{
  return Map<const RowVectorXd>(x, mDimension);
}

MatrixXd buildVectorizationMatrix(const MatrixXd& A)
{
  size_t n = A.cols();

  MatrixXd E = MatrixXd::Identity(n, n);
  MatrixXd S(n*n, n*n);

  for(size_t i = 0; i < n; i++)
    for(size_t j = 0; j < n; j++)
    {
      if(i != j)
        S.block(i*n, j*n, n, n) = A.coeff(i, j)*E;
      else
        S.block(i*n, j*n, n, n) = A + A.coeff(i, j)*E;
    }

  return S;
}

double OptimalControlProblemBase::CalculateCriterionValue(const Eigen::RowVectorXd& theta, int fNumber)
{
  Map<MatrixXd> Y(YVecs[omp_get_thread_num()].data(), A.cols(), A.rows());

  double value = -HUGE_VAL;
  size_t cRows = CMatrices[fNumber].rows();
  for (size_t i = 0; i < cRows; i++)
  {
    RowVectorXd currentVector = CMatrices[fNumber].row(i) + DMatrices[fNumber].row(i)*theta;
    double dotProd = (currentVector*Y).dot(currentVector.transpose());
    value = std::max(value, dotProd);
  }
  return value;
}

double OptimalControlProblemBase::CalculateFunctionals(const double* x, int fNumber)
{
  RowVectorXd theta = getTheta(x);

  if (fNumber == 0)
  {
    MatrixXd Atheta = A + B_u*theta;

    EigenSolver<MatrixXd> eigenSolver(Atheta, false);
    MatrixXcd eigenvalues = eigenSolver.eigenvalues();
    double maxReal = -HUGE_VAL;
    for (int i = 0; i < Atheta.cols(); i++)
      maxReal = std::max(maxReal, eigenvalues.coeff(i).real());

    return maxReal + mZeroConstraintOffset;
  }
  else if (fNumber == 1)
  {
    MatrixXd Atheta = A + B_u*theta;
    MatrixXd S = buildVectorizationMatrix(Atheta);
    MatrixXd rhs = -B_v*B_v.transpose();
    Map<VectorXd> rhsMap(rhs.data(), rhs.size());
    YVecs[omp_get_thread_num()] = S.partialPivLu().solve(rhsMap);
  }

  fNumber--;
  double offset = fNumber == 0 ? -mS : 0.;
  double value = CalculateCriterionValue(theta, fNumber);

  return sqrt(value) + offset;
}
\end{lstlisting}
\begin{lstlisting}[frame=single]
#pragma once
#include "optimal_problem_base.h"

#define _USE_MATH_DEFINES
#include <math.h>
#include <algorithm>
#include <iostream>

using namespace Eigen;
using Eigen::internal::BandMatrix;

class ProblemB_Common : public OptimalControlProblemBase
{
public:
  ProblemB_Common(double S)
  {
    double beta = 0.1;

    int n = 10;

    n_x = 2*n;
    n_v = 1;
    n_u = 1;
    n_k.resize(2);
    n_k[0] = n_k[1] = 1;

    CMatrices.resize(2);
    DMatrices.resize(2);

    A.resize(n_x, n_x);
    A.topLeftCorner(n, n).setZero();
    A.topRightCorner(n, n).setIdentity();

    BandMatrix<double> K(n, n, 1, 1);
    K.diagonal().setConstant(2.);
    K.diagonal(-1).setConstant(-1.);
    K.diagonal(1).setConstant(-1.);
    K.diagonal()(0) = 1.;
    K.diagonal()(n - 1) = 1.;

    A.bottomLeftCorner(n, n) = -K.toDenseMatrix();
    A.bottomRightCorner(n, n) = -beta*K.toDenseMatrix();

    B_v.resize(n_x, 1);
    B_v.col(0).head(n).setZero();
    B_v.col(0).tail(n).setOnes();

    B_u = VectorXd::Zero(n_x);
    B_u(n, 0) = 1.;

    CMatrices[0].resize(1, n_x);
    CMatrices[0].setZero();
    CMatrices[0](0, 0) = 1.;

    CMatrices[1].resize(n - 1, n_x);

    for(size_t i = 0; i < n - 1; i++)
    {
      CMatrices[1].row(i).setZero();
      CMatrices[1].row(i)(i) = -1.;
      CMatrices[1].row(i)(i + 1) = 1.;
    }

    DMatrices[0].resize(1, 1); DMatrices[0] << 0;

    DMatrices[1].resize(n - 1, 1);
    DMatrices[1].setZero();

    mS = S;
    mZeroConstraintOffset = 0.;
  }

  int GetNumberOfConstraints() const { return 2; }
};

class ProblemB : public ProblemB_Common
{
public:
  ProblemB(double S, int dimension) : ProblemB_Common(S)
  {
    mDimension = dimension;//shoud be > 2
  }

  void GetBounds(double* lower, double *upper) const
  {
    for (int i = 0; i < mDimension; i++)
    {
      lower[i] = -10000.;
      upper[i] = -0.01;
    }

    for (int i = 2; i < mDimension; i++)
    {
      upper[i] = 10000.;
    }
  }

  RowVectorXd getTheta(const double* x)
  {
    RowVectorXd theta(n_x);
    theta.setZero();

    theta(0) = x[0] - x[2];
    for (int i = 1; i < mDimension - 2; i++)
        theta(i) = x[i + 1] - x[i + 2];
    theta(mDimension - 2) = x[mDimension - 1];
    theta(n_x / 2) = x[1];

    return theta;
  }
};

class ProblemB2d : public ProblemB_Common
{
public:
  ProblemB2d(double S) : ProblemB_Common(S)
  {
    mDimension = 2;
  }

  void GetBounds(double* lower, double *upper) const
  {
    for (int i = 0; i < mDimension; i++)
    {
      lower[i] = -70.;
      upper[i] = -0.01;
    }
    lower[0] = -20.;
  }

  RowVectorXd getTheta(const double* x)
  {
    RowVectorXd theta(n_x);
    theta.setZero();
    theta(0) = x[0];
    theta(n_x / 2) = x[1];

    return theta;
  }
};
\end{lstlisting}
\begin{lstlisting}[frame=single]
#ifndef __OPTMAL_CONTROL_TEST_IMPL_H__
#define __OPTMAL_CONTROL_TEST_IMPL_H__

#include "optimal_problem_base.h"

class VibroisolationProblem : public OptimalControlProblemBase
{
public:
  VibroisolationProblem()
  {
    n_x = 2;
    n_v = 1;
    n_u = 1;
    n_k.resize(2);
    n_k[0] = n_k[1] = 1;

    CMatrices.resize(2);
    DMatrices.resize(2);

    A.resize(2, 2); A << 0, 1, 0, 0;
    B_u.resize(2, 1); B_u << 0, 1;
    B_v = B_u;

    CMatrices[0].resize(1, 2); CMatrices[0] << 1, 0;
    CMatrices[1].resize(1, 2); CMatrices[1] << 0, 0;

    DMatrices[0].resize(1, 1); DMatrices[0] << 0;
    DMatrices[1].resize(1, 1); DMatrices[1] << 1;
    mDimension = n_x;
    mZeroConstraintOffset = 0.;
    mS = 1.;
  }

  int GetNumberOfConstraints() const { return 2; }

  void GetBounds(double* lower, double *upper) const
  {
    for (int i = 0; i < mDimension; i++)
    {
      upper[i] = -0.2;
      lower[i] = -2.0;
    }
  }
};

class OscillationDampingProblem : public OptimalControlProblemBase
{
public:
  OscillationDampingProblem()
  {
    double beta = 0.1;
    n_x = 4;
    n_v = 1;
    n_u = 1;
    n_k.resize(2);
    n_k[0] = n_k[1] = 1;

    CMatrices.resize(2);
    DMatrices.resize(2);

    A.resize(4, 4);
    A << 0, 0, 1, 0,
      0, 0, 0, 1,
      -2, 1, -2*beta, beta,
      1, -1, beta, -beta;

    B_u.resize(4, 1); B_u << 0, 0, 0, 1;
    B_v.resize(4, 1); B_v << 0, 0, 1, 1;

    CMatrices[0].resize(2, 4);
    CMatrices[0] << 1, 0, 0, 0,
      -1, 1, 0, 0;
    CMatrices[1].resize(1, 4); CMatrices[1] << 0, 0, 0, 0;

    DMatrices[0].resize(2, 1); DMatrices[0] << 0, 0;
    DMatrices[1].resize(1, 1); DMatrices[1] << 1;
    mDimension = n_x;
    mZeroConstraintOffset = 0.02;
    mS = 1.;
  }

  int GetNumberOfConstraints() const
  {
    return 2;
  }

  void GetBounds(double* lower, double *upper) const
  {
    for (int i = 0; i < mDimension; i++)
    {
      upper[i] = 1;
      lower[i] = -2.0;
    }
  }
};
#endif
\end{lstlisting}

\subsection{Приложение 3}
\label{attach3}
\begin{lstlisting}[frame=single]
#ifndef __LOCALMETHOD_H__
#define __LOCALMETHOD_H__

#include "parameters.h"
#include "task.h"
#include "data.h"
#include "common.h"
#include <vector>

#define MAX_LOCAL_TRIALS_NUMBER 10000

class TLocalMethod
{
protected:

  int mDimension;
  int mConstraintsNumber;
  int mTrialsCounter;
  int mMaxTrial;

  TTrial mBestPoint;
  std::vector<TTrial> mSearchSequence;
  TTask* mPTask;

  bool mIsLogPoints;

  double mEps;
  double mStep;
  double mStepMultiplier;

  OBJECTIV_TYPE *mFunctionsArgument;
  OBJECTIV_TYPE *mStartPoint;
  OBJECTIV_TYPE* mCurrentPoint;
  OBJECTIV_TYPE* mCurrentResearchDirection;
  OBJECTIV_TYPE* mPreviousResearchDirection;

  double MakeResearch(OBJECTIV_TYPE*);
  void DoStep();
  double EvaluateObjectiveFunctiuon(const OBJECTIV_TYPE*);

public:

  TLocalMethod();
  TLocalMethod(TParameters _params, TTask* _pTask, TTrial _startPoint, bool logPoints = false);
  ~TLocalMethod();

  void SetEps(double);
  void SetInitialStep(double);
  void SetStepMultiplier(double);
  void SetMaxTrials(int);

  int GetTrialsCounter() const;
  std::vector<TTrial> GetSearchSequence() const;

  TTrial StartOptimization();
};

#endif //__LOCALMETHOD_H__
\end{lstlisting}

\begin{lstlisting}[frame=single]
#include "local_method.h"
#include <cmath>
#include <cstring>
#include <algorithm>

TLocalMethod::TLocalMethod(TParameters _params, TTask* _pTask, TTrial _startPoint, bool logPoints)
{
  mEps = _params.Epsilon / 100;
  if (mEps > 0.0001)
    mEps = 0.0001;
  mBestPoint = _startPoint;
  mStep = _params.Epsilon * 2;
  mStepMultiplier = 2;
  mTrialsCounter = 0;
  mIsLogPoints = logPoints;

  mPTask = _pTask;

  mDimension = mPTask->GetN() - mPTask->GetFixedN();
  mConstraintsNumber = mPTask->GetNumOfFunc() - 1;

  mStartPoint = new OBJECTIV_TYPE[mDimension];
  std::memcpy(mStartPoint,
    _startPoint.y + mPTask->GetFixedN(),
    mDimension * sizeof(OBJECTIV_TYPE));

  mFunctionsArgument = new OBJECTIV_TYPE[mPTask->GetN()];
  std::memcpy(mFunctionsArgument,
    _startPoint.y,
    mPTask->GetN() * sizeof(OBJECTIV_TYPE));
  mMaxTrial = MAX_LOCAL_TRIALS_NUMBER;
}

TLocalMethod::TLocalMethod() : mPTask(NULL), mStartPoint(NULL),
mFunctionsArgument(NULL), mMaxTrial(MAX_LOCAL_TRIALS_NUMBER)
{
}

TLocalMethod::~TLocalMethod()
{
  if (mStartPoint)
    delete[] mStartPoint;
  if (mFunctionsArgument)
    delete[] mFunctionsArgument;
}

void TLocalMethod::DoStep()
{
  for (int i = 0; i < mDimension; i++)
    mCurrentPoint[i] = (1 + mStepMultiplier)*mCurrentResearchDirection[i] -
    mStepMultiplier*mPreviousResearchDirection[i];
}

TTrial TLocalMethod::StartOptimization()
{
  int k = 0, i = 0;
  bool needRestart = true;
  double currentFValue, nextFValue;
  mTrialsCounter = 0;

  mCurrentPoint = new OBJECTIV_TYPE[mDimension];
  mCurrentResearchDirection = new OBJECTIV_TYPE[mDimension];
  mPreviousResearchDirection = new OBJECTIV_TYPE[mDimension];

  while (mTrialsCounter < mMaxTrial) {
    i++;
    if (needRestart) {
      k = 0;
      std::memcpy(mCurrentPoint, mStartPoint, sizeof(OBJECTIV_TYPE)*mDimension);
      std::memcpy(mCurrentResearchDirection, mStartPoint, sizeof(OBJECTIV_TYPE)*mDimension);
      currentFValue = EvaluateObjectiveFunctiuon(mCurrentPoint);
      needRestart = false;
    }

    std::swap(mPreviousResearchDirection, mCurrentResearchDirection);
    std::memcpy(mCurrentResearchDirection, mCurrentPoint, sizeof(OBJECTIV_TYPE)*mDimension);
    nextFValue = MakeResearch(mCurrentResearchDirection);

    if (currentFValue > nextFValue) {
      DoStep();

      if (mIsLogPoints)
      {
        TTrial currentTrial;
        currentTrial.index = mBestPoint.index;
        currentTrial.FuncValues[currentTrial.index] = nextFValue;
        std::memcpy(currentTrial.y, mFunctionsArgument, sizeof(OBJECTIV_TYPE)*mPTask->GetFixedN());
        std::memcpy(currentTrial.y + mPTask->GetFixedN(), mCurrentPoint, sizeof(OBJECTIV_TYPE)*mDimension);
        mSearchSequence.push_back(currentTrial);
      }
      k++;
      currentFValue = nextFValue;
    }
    else if (mStep > mEps) {
      if (k != 0)
        std::memcpy(mStartPoint, mPreviousResearchDirection, sizeof(OBJECTIV_TYPE)*mDimension);
      else
        mStep /= mStepMultiplier;
      needRestart = true;
    }
    else
      break;
  }

  if (currentFValue < mBestPoint.FuncValues[mConstraintsNumber])
  {
    std::memcpy(mBestPoint.y + mPTask->GetFixedN(),
      mPreviousResearchDirection, sizeof(OBJECTIV_TYPE)*mDimension);
    mBestPoint.FuncValues[mConstraintsNumber] = currentFValue;
    mSearchSequence.push_back(mBestPoint);
  }

  delete[] mCurrentPoint;
  delete[] mPreviousResearchDirection;
  delete[] mCurrentResearchDirection;

  return mBestPoint;
}

double TLocalMethod::EvaluateObjectiveFunctiuon(const OBJECTIV_TYPE* x)
{
  if (mTrialsCounter >= mMaxTrial)
    return HUGE_VAL;

  for (int i = 0; i < mDimension; i++)
    if (x[i] < mPTask->GetA()[mPTask->GetFixedN() + i] ||
      x[i] > mPTask->GetB()[mPTask->GetFixedN() + i])
      return HUGE_VAL;

  std::memcpy(mFunctionsArgument + mPTask->GetFixedN(), x, mDimension * sizeof(OBJECTIV_TYPE));
  for (int i = 0; i <= mConstraintsNumber; i++)
  {
    double value = mPTask->GetFuncs()[i](mFunctionsArgument);
    if (i < mConstraintsNumber && value > 0)
    {
      mTrialsCounter++;
      return HUGE_VAL;
    }
    else if (i == mConstraintsNumber)
    {
      mTrialsCounter++;
      return value;
    }
  }

  return HUGE_VAL;
}

void TLocalMethod::SetEps(double eps)
{
  mEps = eps;
}

void TLocalMethod::SetInitialStep(double value)
{
  mStep = value;
}

void TLocalMethod::SetStepMultiplier(double value)
{
  mStepMultiplier = value;
}

void TLocalMethod::SetMaxTrials(int count)
{
  mMaxTrial = std::min(count, MAX_LOCAL_TRIALS_NUMBER);
}

int TLocalMethod::GetTrialsCounter() const
{
  return mTrialsCounter;
}

std::vector<TTrial> TLocalMethod::GetSearchSequence() const
{
  return mSearchSequence;
}

double TLocalMethod::MakeResearch(OBJECTIV_TYPE* startPoint)
{
  double bestValue = EvaluateObjectiveFunctiuon(startPoint);

  for (int i = 0; i < mDimension; i++)
  {
    startPoint[i] += mStep;
    double rightFvalue = EvaluateObjectiveFunctiuon(startPoint);

    if (rightFvalue > bestValue)
    {
      startPoint[i] -= 2 * mStep;
      double leftFValue = EvaluateObjectiveFunctiuon(startPoint);
      if (leftFValue > bestValue)
        startPoint[i] += mStep;
      else
        bestValue = leftFValue;
    }
    else
      bestValue = rightFvalue;
  }

  return bestValue;
}
\end{lstlisting}

\subsection{Приложение 4}
\label{attach4}
\begin{lstlisting}[frame=single]
#ifndef __DUAL_QUEUE_H__
#define __DUAL_QUEUE_H__

#include "minmaxheap.h"
#include "common.h"
#include "queue_common.h"

class TPriorityDualQueue : public TPriorityQueueCommon
{
protected:
  int MaxSize;
  int CurLocalSize;
  int CurGlobalSize;

  MinMaxHeap< TQueueElement, _less >* pGlobalHeap;
  MinMaxHeap< TQueueElement, _less >* pLocalHeap;

  void DeleteMinLocalElem();
  void DeleteMinGlobalElem();
  void ClearLocal();
  void ClearGlobal();
public:

  TPriorityDualQueue(int _MaxSize = DefaultQueueSize); // _MaxSize must be qual to 2^k - 1
  ~TPriorityDualQueue();

  int GetLocalSize() const;
  int GetSize() const;
  int GetMaxSize() const;
  bool IsLocalEmpty() const;
  bool IsLocalFull() const;
  bool IsEmpty() const;
  bool IsFull() const;

  void Push(double globalKey, double localKey, void *value);
  void PushWithPriority(double globalKey, double localKey, void *value);
  void Pop(double *key, void **value);

  void PopFromLocal(double *key, void **value);

  void Clear();
  void Resize(int size);
};
#endif
\end{lstlisting}

\begin{lstlisting}[frame=single]
#include "dual_queue.h"

TPriorityDualQueue::TPriorityDualQueue(int _MaxSize)
{
  MaxSize = _MaxSize;
  CurGlobalSize = CurLocalSize = 0;
  pLocalHeap = new MinMaxHeap< TQueueElement, _less >(MaxSize);
  pGlobalHeap = new MinMaxHeap< TQueueElement, _less >(MaxSize);
}

TPriorityDualQueue::~TPriorityDualQueue()
{
  delete pLocalHeap;
  delete pGlobalHeap;
}

void TPriorityDualQueue::DeleteMinLocalElem()
{
  TQueueElement tmp = pLocalHeap->popMin();
  CurLocalSize--;

  //update linked element in the global queue
  if (tmp.pLinkedElement != NULL)
    tmp.pLinkedElement->pLinkedElement = NULL;
}

void TPriorityDualQueue::DeleteMinGlobalElem()
{
  TQueueElement tmp = pGlobalHeap->popMin();
  CurGlobalSize--;

  //update linked element in the local queue
  if (tmp.pLinkedElement != NULL)
    tmp.pLinkedElement->pLinkedElement = NULL;
}

int TPriorityDualQueue::GetLocalSize() const
{
  return CurLocalSize;
}

int TPriorityDualQueue::GetSize() const
{
  return CurGlobalSize;
}

int TPriorityDualQueue::GetMaxSize() const
{
  return MaxSize;
}

bool TPriorityDualQueue::IsLocalEmpty() const
{
  return CurLocalSize == 0;
}

bool TPriorityDualQueue::IsLocalFull() const
{
  return CurLocalSize == MaxSize;
}

bool TPriorityDualQueue::IsEmpty() const
{
  return CurGlobalSize == 0;
}

bool TPriorityDualQueue::IsFull() const
{
  return CurGlobalSize == MaxSize;
}

void TPriorityDualQueue::Push(double globalKey, double localKey, void * value)
{
  TQueueElement* pGlobalElem = NULL, *pLocalElem = NULL;
  //push to a global queue
  if (!IsFull()) {
    CurGlobalSize++;
    pGlobalElem = pGlobalHeap->push(TQueueElement(globalKey, value));
  }
  else {
    if (globalKey > pGlobalHeap->findMin().Key) {
      DeleteMinGlobalElem();
      CurGlobalSize++;
      pGlobalElem = pGlobalHeap->push(TQueueElement(globalKey, value));
    }
  }
  //push to a local queue
  if (!IsLocalFull()) {
    CurLocalSize++;
    pLocalElem = pLocalHeap->push(TQueueElement(localKey, value));
  }
  else {
    if (localKey > pLocalHeap->findMin().Key) {
      DeleteMinLocalElem();
      CurLocalSize++;
      pLocalElem = pLocalHeap->push(TQueueElement(localKey, value));
    }
  }
  //link elements
  if (pGlobalElem != NULL && pLocalElem != NULL) {
    pGlobalElem->pLinkedElement = pLocalElem;
    pLocalElem->pLinkedElement = pGlobalElem;
  }
}

void TPriorityDualQueue::PushWithPriority(double globalKey, double localKey, void * value)
{
  TQueueElement* pGlobalElem = NULL, *pLocalElem = NULL;
  //push to a global queue
  if (!IsEmpty()) {
    if (globalKey > pGlobalHeap->findMin().Key) {
      if (IsFull())
        DeleteMinGlobalElem();
      CurGlobalSize++;
      pGlobalElem = pGlobalHeap->push(TQueueElement(globalKey, value));
    }
  }
  else {
    CurGlobalSize++;
    pGlobalElem = pGlobalHeap->push(TQueueElement(globalKey, value));
  }
  //push to a local queue
  if (!IsLocalEmpty()) {
    if (localKey > pLocalHeap->findMin().Key) {
      if (IsLocalFull())
        DeleteMinLocalElem();
      CurLocalSize++;
      pLocalElem = pLocalHeap->push(TQueueElement(localKey, value));
    }
  }
  else {
    CurLocalSize++;
    pLocalElem = pLocalHeap->push(TQueueElement(localKey, value));
  }
  //link elements
  if (pGlobalElem != NULL && pLocalElem != NULL) {
    pGlobalElem->pLinkedElement = pLocalElem;
    pLocalElem->pLinkedElement = pGlobalElem;
  }
}

void TPriorityDualQueue::PopFromLocal(double * key, void ** value)
{
  TQueueElement tmp = pLocalHeap->popMax();
  *key = tmp.Key;
  *value = tmp.pValue;
  CurLocalSize--;

  //delete linked element from the global queue
  if (tmp.pLinkedElement != NULL) {
    pGlobalHeap->deleteElement(tmp.pLinkedElement);
    CurGlobalSize--;
  }
}

void TPriorityDualQueue::Pop(double * key, void ** value)
{
  TQueueElement tmp = pGlobalHeap->popMax();
  *key = tmp.Key;
  *value = tmp.pValue;
  CurGlobalSize--;

  //delete linked element from the local queue
  if (tmp.pLinkedElement != NULL)	{
    pLocalHeap->deleteElement(tmp.pLinkedElement);
    CurLocalSize--;
  }
}

void TPriorityDualQueue::Clear()
{
  ClearLocal();
  ClearGlobal();
}

void TPriorityDualQueue::Resize(int size)
{
  MaxSize = size;
  CurGlobalSize = CurLocalSize = 0;
  delete pLocalHeap;
  delete pGlobalHeap;
  pLocalHeap = new MinMaxHeap< TQueueElement, _less >(MaxSize);
  pGlobalHeap = new MinMaxHeap< TQueueElement, _less >(MaxSize);
}

void TPriorityDualQueue::ClearLocal()
{
  pLocalHeap->clear();
  CurLocalSize = 0;
}

void TPriorityDualQueue::ClearGlobal()
{
  pGlobalHeap->clear();
  CurGlobalSize = 0;
}
\end{lstlisting}
